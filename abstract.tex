Multithreaded programming is here to stay, and concurrency bugs are the focus of a growing number of attacks.
While most defensive efforts against such attacks seek to identify bugs during debugging, an alternative method seeks to make exploitation harder without the need to first identify the bugs - or even the fact that there are any.
Time randomization introduces more diversity among instances of the same software.
Because concurrency attacks inherently depend on the timing between and among threads, diversity in thread timing may decrease the chance that a successful attack on one system will succeed on another.

We study three implementations of time randomization, all using the injection of NOPs to alter program timing.
The application of these implementations to two real-world concurrency bugs results in a marked increase in the cost to exploit those bugs.
After demonstrating the effectiveness of the method, especially when NOPs are injected before library function calls following synchronization points, methods for improving the efficiency of this defense against concurrency attacks in future research are proposed.