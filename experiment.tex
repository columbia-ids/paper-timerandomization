To test both the efficacy and the performance of time randomization as a mechanism to thwart concurrency bug exploitation, we applied time randomization to two canonical and two real concurrency bugs.
\subsection{Real Bug: Libsafe CVE-2005-1125}
Libsafe~\cite{Tsai2001} is a library which protects processes against the
exploitation of buffer overflow vulnerabilities in process stacks, as well as
format string vulnerabilities.  It does this by intercepting all calls to C
standard library functions known to be vulnerable, and then running ``safe'' versions of those functions which issue warnings about exploit attempts before exiting without allowing the exploits to succeed.

A static global variable `dying' is used in Libsafe to indicate when an unsafe action has already been detected, and Libsafe is issuing warnings and exiting.  In this case, Libsafe stops checking for new exploit attempts.  However, this variable is not protected by any synchronization mechanisms, so in the case of a multithreaded program, the following sequence of events is possible:
\begin{enumerate}
	\item Thread A attempts an exploit.
	\item The exploit is caught by Libsafe, the `dying' flag is set, and Libsafe begins the warning and exit procedure.
	\item Thread B is scheduled, and attempts another exploit before Libsafe has exited thread A.
	\item Because the `dying' flag is set, thread B's exploit is not caught by Libsafe, and it succeeds.
\end{enumerate}
The proof-of-concept exploit (Figure \ref{fig_poc})\cite{Bugtraq13190} often realizes the above sequence of events, effectively bypassing Libsafe.
\begin{figure}
	\lstinputlisting{libsafe-PoC.c}
	\caption{The proof of concept exploit for concurrency bug CVE-2005-1125~\cite{CVE-2005-1125}.  Both functions \texttt{func1} and \texttt{func2} attempt to overflow buffers in their own stack frames.  Either function running alone would be caught by Libsafe at the illegal calls to \texttt{strcpy}, but when run in parallel, the concurrency bug may allow one call to \texttt{strcpy} to succeed.}
	\label{fig_poc}
\end{figure}

Time randomization was applied to Libsafe as described in \autoref{implementation}.
The time required to make a (legal) \texttt{strcpy} function call (using Libsafe and averaged over 10,000,000 \texttt{strcpy} calls), and the proof of concept exploit success rate were measured for several randomizations for each max delay, where max delay was varied from 0 to 50,000.
\textbf{T2} and \textbf{T3} were special cases for this bug in that only one external library function was identified as immediately preceding a synchronization mechanism and only one external library function was identified as immediately following a synchronization mechanism.
In these cases, we were able to exactly characterize the effects of NOP injection as a function of the number of NOPs injected into that function.
\subsection{Real Bug: Libvirt CVE-2014-1447}
Libvirt~\cite{libvirt} is a toolkit for interacting with virtual machines and hypervisors.
This toolkit consists, in part, of a daemon `libvirtd' which listens for and
responds to virtual machine management requests (e.g., to connect to a remote host, or shutdown a VM).

Prior to version 1.2.1, the Libvirt daemon suffered from concurrency bug CVE-2014-1447 which allows an unauthenticated client to crash the daemon with seemingly innocuous requests to connect to a host.
After receiving a request from a client to open a connection, the daemon spawns a thread to open and maintain that connection.
The spawned thread will eventually dereference a pointer called
\texttt{client->keepalive} as part of opening the connection.
However, during this vulnerability window, the client may close the
connection, which the daemon will handle in the main thread by (among other
things) setting \texttt{client->keepalive} to \texttt{NULL}.
Then when the spawned thread dereferences this pointer, the daemon crashes.~\cite{RHELbug1047577}

The Libvirt developers' proof of concept makes two changes to Libvirt in order to reliably reproduce this bug.
First, the client is modified to exit as soon as it has requested a connection.
Second, the daemon is modified to sleep in threads spawned to open new
connections, before dereferencing the \texttt{client->keepalive} pointer.
The first modification is within an attacker's ability, in our threat model.
However, the second modification requires control of the daemon, which we assume the attacker does not yet have.
An alternate exploit requiring only the first modification was applied to Libvirt 0.9.8.
In this alternate exploit, the client opens (and closes) as many connections as it can until the daemon crashes.
Without modification to the daemon, this alternate exploit reliably reproduced the bug and crashed the daemon.

Time randomization was applied to the Libvirt daemon as described in \autoref{implementation}.
The time required to open and close a connection properly (with the unmodified client averaged over 10,000 sequential connections) was measured, and the time required for the alternate exploit described above to crash the daemon was measured five times for a single randomization for each max delay, where max delay was varied from 0 to 1,000,000.
All exploit attempts were interrupted and considered failed attempts after ten minutes if they did not succeed before then.
\subsection{Canonical Bug: Nonatomic Operations Assumed to be Atomic}
Farchi et al. presented a partial concurrency bug taxonomy~\cite{Farchi2003}, and Lu et al. later showed that in a narrow study of real world concurrency bugs, the vast majority are one of two types~\cite{Lu2008}:
\begin{enumerate}
	\item Atomicity Violation: The desired serializability among multiple memory accesses is violated.
	\item Order Violation: The desired order between two memory accesses is flipped.
\end{enumerate}
To test time randomization on the first type, a simple concurrency bug was
constructed mimicking a ``nonatomic operations assumed to be atomic'' programmer error~\cite{Farchi2003}.
In particular, in Python, the statement \texttt{x+=1} is nonatomic, but may be mistaken to be atomic due to its compact syntax.
Our simple concurrency bug (Figure \ref{fig_nonatomic}) contains a loop that expects this statement (unprotected by synchronization mechanisms) to always increment \texttt{x} by one.
An ``exploit'' runs in a separate thread, modifying \texttt{x} periodically.
The exploit succeeds when the statement \texttt{x+=1} in the first thread fails to increment \texttt{x} by exactly one.
In other words, the exploit thread modifies \texttt{x} within the vulnerability window (between when \texttt{x} is read from memory, and when it is written back to memory by the first thread).
\begin{figure}
	\lstinputlisting{nonatomic.py}
	\caption{A canonical atomicity violation concurrency bug.  The \texttt{xplusplus} function assumes the nonatomic statement \texttt{x+=1} to be atomic.  The \texttt{fuzzer} function, run in a separate thread, attempts to exploit this assumption by modifying \texttt{x} within the vulnerability window (between when \texttt{x} is read from memory, and when it is written back to memory).}
	\label{fig_nonatomic}
\end{figure}

Time randomization was applied to this bug as described in \autoref{implementation}.
The time required to complete the operations encoded in the statement \texttt{x+=1} (averaged over 10,000,000 sequential executions of that statement), and the number of exploit loop iterations required to succeed were measured for a single randomization for each max delay, where max delay was varied from 0 to 1,000,000.
%TODO: discuss sleep in exploit thread
%Knowledge of the time required to execute the instructions encoded by the
%statement \texttt{x+=1} can be utilized by the exploit thread to achieve
%``exploitation'' in fewer loops.
%Note in Figure \ref{fig_nonatomic} that the exploit thread is identical to the buggy thread except for a \texttt{sleep()} at the beginning of each loop iteration.
%The \texttt{sleep()} has a duration of half of the time required to complete the statement \texttt{x+=1}.
%This sleep can be thought of as shifting the exploit thread loop (\texttt{fuzz}) with respect to the buggy thread loop (\texttt{xplusplus}).
%In the best case (for the exploit), this corresponds to altering \texttt{x} at intervals of half of the vulnerability window (modulo the time for one loop iteration).
%Of course the shift is imperfect due to the nondeterminism in thread scheduling, but the effect is significant in practice.
%The table below shows the average number of `exploit loop' iterations required to alter \texttt{x} within the vulnerability window over ten runs.
%This measurement is repeated for values of \texttt{HALF} equal to (0, 2.8e-05, 2.8e-04, 2.8e-03, and 2.8e-02), where the measured time to complete the statement \texttt{x+=1} on the test system was approximately
%\subsection{Canonical Bug: Order Violation}
%TODO