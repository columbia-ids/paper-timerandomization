To test both the efficacy and the performance of time randomization as a mechanism to thwart concurrency bug exploitation, we applied time randomization to two real concurrency bugs.
\subsection{Libsafe CVE-2005-1125}
Libsafe~\cite{Tsai2001} is a library which protects processes against the
exploitation of buffer overflow vulnerabilities in process stacks, as well as
format string vulnerabilities.  It does this by intercepting all calls to C
standard library functions known to be vulnerable, and then running ``safe'' versions of those functions which issue warnings about exploit attempts before exiting without allowing the exploits to succeed.

A static global variable `dying' is used in Libsafe to indicate when an unsafe action has already been detected, and Libsafe is issuing warnings and exiting.  In this case, Libsafe stops checking for new exploit attempts.  However, this variable is not protected by any synchronization mechanisms, so in the case of a multithreaded program, the following sequence of events is possible:
\begin{enumerate}
	\item Thread A attempts an exploit.
	\item The exploit is caught by Libsafe, the `dying' flag is set, and Libsafe begins the warning and exit procedure.
	\item Thread B is scheduled, and attempts another exploit before Libsafe has exited thread A.
	\item Because the `dying' flag is set, thread B's exploit is not caught by Libsafe, and it succeeds.
\end{enumerate}
The proof-of-concept exploit~\cite{Bugtraq13190} often realizes the above sequence of events, effectively bypassing Libsafe.

Time randomization was applied to Libsafe as described in \autoref{implementation}.
The time required to make a (legal) \texttt{strcpy} function call (using Libsafe and averaged over 10,000,000 \texttt{strcpy} calls), and the proof of concept exploit success rate were measured for several randomizations for each max delay, where max delay was varied from 0 to 50,000.
Recall that the max delay is used as the upper bound on the range from which the NOP loop length for each function interposition is randomly chosen, as described in \autoref{implementation}.
\textbf{T2} and \textbf{T3} were special cases for this bug in that only one external library function was identified as immediately preceding a synchronization mechanism and only one external library function was identified as immediately following a synchronization mechanism.
In these cases, we were able to exactly characterize the effects of NOP injection as a function of the number of NOPs injected into that function.
\subsection{Libvirt CVE-2014-1447}
Libvirt~\cite{libvirt} is a toolkit for interacting with virtual machines and hypervisors.
This toolkit consists, in part, of a daemon `libvirtd' which listens for and
responds to virtual machine management requests (e.g., to connect to a remote host, or shutdown a VM).

Prior to version 1.2.1, the Libvirt daemon suffered from concurrency bug CVE-2014-1447 which allows an unauthenticated client to crash the daemon with seemingly innocuous requests to connect to a host.
After receiving a request from a client to open a connection, the daemon spawns a thread to open and maintain that connection.
The spawned thread will eventually dereference a pointer called
\texttt{client->keepalive} as part of opening the connection.
However, during this vulnerability window, the client may close the
connection, which the daemon will handle in the main thread by (among other
things) setting \texttt{client->keepalive} to \texttt{NULL}.
Then when the spawned thread dereferences this pointer, the daemon crashes.~\cite{RHELbug1047577}

The Libvirt developers' proof of concept makes two changes to Libvirt in order to reliably reproduce this bug.
First, the client is modified to exit as soon as it has requested a connection.
Second, the daemon is modified to sleep in threads spawned to open new
connections, before dereferencing the \texttt{client->keepalive} pointer.
The first modification is within an attacker's ability, in our threat model.
However, the second modification requires control of the daemon, which we assume the attacker does not yet have.
An alternate exploit requiring only the first modification was applied to Libvirt 0.9.8.
In this alternate exploit, the client opens (and closes) as many connections as it can until the daemon crashes.
Without modification to the daemon, this alternate exploit reliably reproduced the bug and crashed the daemon.

Time randomization was applied to the Libvirt daemon as described in \autoref{implementation}.
The time required to open and close a connection properly (with the unmodified client averaged over 10,000 sequential connections) was measured, and the time required for the alternate exploit described above to crash the daemon was measured five times for a single randomization for each max delay.
For \textbf{T1}, the max delay was varied from 0 to 1,000.
For \textbf{T2} and \textbf{T3}, the max delay was varied from 0 to 1,000,000.
All exploit attempts were interrupted and considered failed attempts after ten minutes if they did not succeed before then.
